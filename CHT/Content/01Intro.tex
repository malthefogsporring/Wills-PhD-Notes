\newpage
\section{Introduction}
\quote{``Desar's chosen field in mathematics was so esoteric that nobody in the institute or the Maths Federative could really check his progress"
\\\textit{The Dispossessed, Ursula K. Le Guin}}
\vspace{2ex}


The chromatic filtration on the stable homotopy groups of spheres is a reflection of the height filtration on the moduli stack of commutative, 1-dimensional formal group laws. Taking this idea seriously means forging a link between arithmetic geometry and homotopy theory. Periodicity phenomena deep within stable homotopy theory reflect structures found on the moduli stack of formal groups (or, perhaps better, p-divisible groups).

Many special objects in mathematics meet in chromatic homotopy theory: complex orientations, the Adams-Novikov spectral sequence, formal groups, p-divisible groups, modular forms, and even (conjecturally) certain quantum field theories. One theme I'll be particularly enthused about is using the Fargues-Fontaine curve to reimagine the geometry controlling the chromatic filtration. 

