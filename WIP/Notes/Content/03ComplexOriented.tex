\newpage
\section{Our first introduction to Spectra}

\subsection{What are Spectra?}
\subsubsection{Homology and Spectra}
% Ordinary cohomology + K-theory, E_\infty rings
\subsubsection{Group completion and the Barrat-Priddy-Quillen Theorem}



\newpage
\subsection{Thom Spectra and Complex Cobordism}
We want to focus on a particularly interesting spectrum; that of complex cobordism $\MU$. We will construct $\MU$ in the usual way, via Thom spectra, following the more modern account of (\cite{Ando2008}) and (\cite{Ando2013}) to achieve this.

As described in \cite{Ando2008}, Thom spectra define the origin of $\mathbb{E}_\infty$-rings and the accompanying phrase `brave new algebras', where it was realised that units of ring-spectra play an essential role in obstruction theory; the Thom isomorphism $\Phi : H^k(B; \Z/2\Z) \cong \tilde{H}^{k+n}(T(E),\Z/2\Z)$ gives that, since Stiefel-Whitney classes can be generated via the Steenrod operations (namely as $w_i(p) = \Phi^{-1}(Sq^i(\Phi(1)))$) they are stable homotopy invariants. This isn't true for other characteristic classes, and since they give an obstruction to finding a linearly independent set of sections of a bundle (i.e. an orientation) they give a homological approach to studying sections. So we aim to follow the analogy of Steifel-Whitney classes, to whit we must understand analogies of principal bundles, sections and Thom spaces.

We can think of a vector bundle $E \to X$ as a space $X$ paremetrising some vectorspaces $\{E_x \,:\, x \in X\}$ in some interesting way, which is what we want to emulate now. Our first analogy is that of replacing vectorspaces with $R$-modules for some ring spectrum $R$. We should understand locally free rank-one modules as modules $L$ with a specified equivalece $L \xrightarrow{\sim} R$, which will play the role of sections. Letting $R\text{Line} \subseteq R\text{Mod}$ denote the subcategory of locally free rank-one bundles sitting inside the category of $R$-modules we may define a \textit{bundle of $R$-modules} over a space $X$ as a map\footnote{We note here that straightening-unstraightening gives us the more intuitive notion of a bundle as an element of $R\text{Mod}/X$, but ths definition is more useful.} $X^\text{op} \to R\text{Mod}$. We want to know to what extent we can understand this bundle of $R$-modules as a collection of `sections', i.e. maps $X^\text{op} \to R\text{Line}$, and the linear-independence can be understood as an `optimal' set of such sections, i.e. we want to recover the bundle from the left Kan extension
	\begin{center}
	\begin{tikzcd}
	R\text{Line}			\arrow[r,"\iota"]\arrow[d,"\yo_X"']
	& R\text{Mod}
	\\
	\Fun(X^\text{op},R\text{Line})	\arrow[ur,dashed,"M"']
	\end{tikzcd}
	\end{center}
where $\yo_X$ is the yoneda embedding $\yo_X(Y) := \Fun(X^\text{op},Y)$ and $\iota$ is the inclusion. Using the colimit formula for left Kan extensions (since $R\text{Line}$ is small and $R\Mod$ is cocomplete) this becomes 
	\begin{equation}\label{eqn:thom-bundle}
		M(f)
		= \underset{\eta : \yo_X L \to f}{\colim} \iota(L)
		= \colim \big( X^\text{op} \xrightarrow{f} R\text{Line} \to R\Mod  \big).
	\end{equation}
If we understand the bundle $\yo_XL$ as the trivial bundle then this is exactly seen as gluing trivial bundles together to get $M(f)$, just as in the classical story with local charts for principal bundles.
	\begin{definition}[Thom Spectrum]{def:thom-spectrum}
	We call $M(f)$ the \textbf{Thom spectrum} associated to the bundle $f : X^\text{op} \to R\text{Line}$.
	\end{definition}








%section should be bundles of R-lines

%%% The colimit formula in (Ando et al, 2013) looks like a Kan extension??









\newpage
\subsection{Complex Oriented Cohomology Theories}



\newpage
\subsection{Goerss-Hopkins obstruction theory}
Our main reference for this section will be (\cite{Mazel-Gee2018}).

%\newpage
%\subsection{MU and Formal Group Laws}
%\newpage
%\subsection{BP and Morava K-theory}



